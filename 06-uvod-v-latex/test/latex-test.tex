\documentclass{article}
\begin{document}
\section*{Pitagorov izrek}
V pravokotnem trikotniku s katetama \(a\) in \(b\) ter hipotenuzo \(c\) velja
\[ a^2 + b^2 = c^2 \]
\section*{izraz za izračun hipotenuze}
Iz zgornjega izreka sledi, da lahko hipotenuzo izračunamo po formuli
\[ c = \sqrt{a^2 + b^2} \]
\end{document}